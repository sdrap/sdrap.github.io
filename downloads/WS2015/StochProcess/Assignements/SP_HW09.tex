%-----------------------   general info   ----------------------------------------	

%	filename  	= 	homework_template
%   date      	= 	
%   time      	= 	
%   author    	= 	Samuel Drapeau (Based on a template provided by Daniel, University of Constance)
% 	adress		= 	Shanghai Jiao Tong University

%-----------------------   documentclass, packages   -----------------------------
\documentclass[DIV=classic,a4paper,10pt]{scrartcl}

\KOMAoptions{DIV=last}

\setkomafont{title}{\scshape}
\setkomafont{disposition}{\normalcolor\scshape}

%\setkomafont{title}{\bfseries}
%\setkomafont{disposition}{\normalfont\normalcolor\bfseries}


%----------------------- general packages (fonts, language, enumitem) ------------
\usepackage{times}
\usepackage[english]{babel}
\usepackage[T1]{fontenc}
\usepackage[latin1]{inputenc}
\usepackage{enumitem}
%\usepackage{fullpage}



%----------------------- Math Packages -------------------------------------------

\usepackage{amsfonts}
\usepackage{amssymb}
\usepackage[intlimits]{amsmath}
\usepackage[hyperref,amsmath,thmmarks]{ntheorem}
\usepackage{stmaryrd}
%\parindent0cm

%-----------------------   math operators   --------------------------------------	
%-----------------------   general short cuts   ----------------------------------	

%-----------------------   theorem environments   --------------------------------


\theoremseparator{.}
\newtheorem{theorem}{Theorem}[section]
\newtheorem{corollary}[theorem]{Corollary}
\newtheorem{conjecture}[theorem]{Conjecture}
\newtheorem{assumption}[theorem]{Assumption}
\newtheorem{lemma}[theorem]{Lemma}
\newtheorem{proposition}[theorem]{Proposition}
\newtheorem{exercise}[theorem]{Exercise}

\theorembodyfont{\upshape}
\newtheorem{definition}[theorem]{Definition}

\theoremsymbol{\ensuremath{\lozenge}}
\newtheorem{example}[theorem]{Example}

\theoremsymbol{\ensuremath{\blacklozenge}}
\theoremheaderfont{\itshape}
\newtheorem{remark}[theorem]{Remark}

\theoremsymbol{\ensuremath{\blacklozenge}}
\theoremheaderfont{\itshape}
\newtheorem{remarks}[theorem]{Remarks}

\theoremsymbol{\ensuremath{\square}}
\theoremheaderfont{\itshape}
\theoremstyle{nonumberplain}
\newtheorem{proof}{Proof}



%---------------------------------------------------------------------------------
% \numberwithin{section}{chapter}\numberwithin{equation}{chapter}
\numberwithin{equation}{section}
\setcounter{tocdepth}{2}


\begin{document}

\noindent
Teacher: Samuel Drapeau \hfill Shanghai Jiao Tong University \newline
Teaching Assistant: Zhang Yaoyuan \hfill WS 2015/2016

\smallskip
\noindent
\hrulefill

\smallskip
%-----------------------   mainmatter   ------------------------------------------

\setcounter{section}{9}

\pagestyle{empty}


%-----------------------------------------------------------------------
\section*{``Stochastic Processes'' -- Homework Sheet 9}
\thispagestyle{empty}


%-----------------------------------------------------------------------


\begin{exercise}(5 points)
    Let $X$ and $Y$ be two integrable random variables independent of each other and identically distributed.
    Show that
    \begin{equation*}
        E\left[ X |X+Y \right]=\frac{X+Y}{2}
    \end{equation*}
\end{exercise}

\begin{exercise}(Bonus 5 points)
    Let $\mathcal{G}_1$ and $\mathcal{G}_2$ be two $\sigma$-algebra independent of each other.
    Let $X$ be an integrable random variable such that $\sigma(\sigma(X),\mathcal{G}_1)$ is independent of $\mathcal{G}_2$.
    Show that
    \begin{equation*}
        E\left[ X |\sigma(\mathcal{G}_1,\mathcal{G}_2)) \right]=E\left[ X|\mathcal{G}_1 \right]
    \end{equation*}
\end{exercise}


\begin{exercise}(5 points)
    Let $X$ be a positive random variable.
    Show that
    \begin{equation*}
        E\left[ X \right]=\int_{0}^{\infty}P\left[ X > x \right] dx
    \end{equation*}
\end{exercise}



\begin{exercise}(20 points)

    Let $X$ and $Y$ be two random variable with a joint distribution $P_{(X,Y)}$ absolute continuous with respect to the Lebesgue measure on $\mathbb{R}^2$.
    Show that
    \begin{itemize}
        \item show that $P_{(X,Y)}$ has a density, that it, there exists a measurable function $f_{(X,Y)}:\mathbb{R}^2\to \mathbb{R}_+$, such that
            \begin{equation*}
                E\left[ g(X,Y) \right]=\int_{\mathbb{R}^2} g(x,y)f_{(X,Y)}(x,y) dx dy
            \end{equation*}
            for every positive measurable function $g:\mathbb{R}^2\to [0,\infty[$.
        \item show that $P_X$ as well as $P_Y$ are also absolutely continuous with respect to Lebegue.
            Provide an expression for their density $f_X$ and $f_Y$ respectively.
        \item show that $P_{(X,Y)}=P_X\otimes K$ for some stochastic kernel
            \begin{equation*}
                K(x,B)=\int_{B}^{} f_{(Y|X)}(x,y)dy
            \end{equation*}
            where $f_{(Y|X)}:\mathbb{R}^2\to \mathbb{R}$ is to be determined as a function of $f_X$, $f_Y$ and $f_{(X,Y)}$.
        \item show that 
            \begin{equation*}
                E\left[ g(X,Y) | X\right]=\int_{\mathbb{R}^2}^{} f_{Y|X}(X,y) dy
            \end{equation*}
            for every positive measurable random variable $g:\mathbb{R}^2\to \mathbb{R}$.
        \item Suppose that $f_{(X,Y)}=21_{[0,1]}(x+y)1_{[0\infty[}(x)1_{[0,\infty[}(y)$.
            Compute $E[\exp(Y)|X]$ explicitly.
    \end{itemize}
\end{exercise}

\begin{exercise}(5 points)
    Let $(X_n)$ be a sequence of independent random variables such that each $X_n$ is $\mathcal{N}(\mu,\sigma)$-distributed with $\mu\in \mathbb{R}$ and $\sigma>0$, that is, they are all identically distributed with a probability density function with respect to Lebesgue given by
    \begin{equation*}
        f_{X_n}(x)=f_{X_1}(x)=\frac{1}{\sigma \sqrt{2\pi}}\exp\left( -\frac{(x-\mu)^2}{2\sigma^2} \right)
    \end{equation*}
    We define the process $M$ as
    \begin{equation*}
        M_0=1, \quad \text{and}\quad M_t=\exp\left( \sum_{s=1}^t X_s-\frac{n\sigma^2}{2} \right)
    \end{equation*}
    Show that $M$ is a converging martingale and compute its limit.
\end{exercise}




%\begin{proof}

%\end{proof}


\smallskip
\noindent
\textbf{Due date:} Upload before Monday 2015.11.30 14:00.

\end{document}
