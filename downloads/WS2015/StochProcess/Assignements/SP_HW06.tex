%-----------------------   general info   ----------------------------------------	

%	filename  	= 	homework_template
%   date      	= 	
%   time      	= 	
%   author    	= 	Samuel Drapeau (Based on a template provided by Daniel, University of Constance)
% 	adress		= 	Shanghai Jiao Tong University

%-----------------------   documentclass, packages   -----------------------------
\documentclass[DIV=classic,a4paper,10pt]{scrartcl}

\KOMAoptions{DIV=last}

\setkomafont{title}{\scshape}
\setkomafont{disposition}{\normalcolor\scshape}

%\setkomafont{title}{\bfseries}
%\setkomafont{disposition}{\normalfont\normalcolor\bfseries}


%----------------------- general packages (fonts, language, enumitem) ------------
\usepackage{times}
\usepackage[english]{babel}
\usepackage[T1]{fontenc}
\usepackage[latin1]{inputenc}
\usepackage{enumitem}
%\usepackage{fullpage}



%----------------------- Math Packages -------------------------------------------

\usepackage{amsfonts}
\usepackage{amssymb}
\usepackage[intlimits]{amsmath}
\usepackage[hyperref,amsmath,thmmarks]{ntheorem}
\usepackage{stmaryrd}
%\parindent0cm

%-----------------------   math operators   --------------------------------------	
%-----------------------   general short cuts   ----------------------------------	

%-----------------------   theorem environments   --------------------------------


\theoremseparator{.}
\newtheorem{theorem}{Theorem}[section]
\newtheorem{corollary}[theorem]{Corollary}
\newtheorem{conjecture}[theorem]{Conjecture}
\newtheorem{assumption}[theorem]{Assumption}
\newtheorem{lemma}[theorem]{Lemma}
\newtheorem{proposition}[theorem]{Proposition}
\newtheorem{exercise}[theorem]{Exercise}

\theorembodyfont{\upshape}
\newtheorem{definition}[theorem]{Definition}

\theoremsymbol{\ensuremath{\lozenge}}
\newtheorem{example}[theorem]{Example}

\theoremsymbol{\ensuremath{\blacklozenge}}
\theoremheaderfont{\itshape}
\newtheorem{remark}[theorem]{Remark}

\theoremsymbol{\ensuremath{\blacklozenge}}
\theoremheaderfont{\itshape}
\newtheorem{remarks}[theorem]{Remarks}

\theoremsymbol{\ensuremath{\square}}
\theoremheaderfont{\itshape}
\theoremstyle{nonumberplain}
\newtheorem{proof}{Proof}



%---------------------------------------------------------------------------------
% \numberwithin{section}{chapter}\numberwithin{equation}{chapter}
\numberwithin{equation}{section}
\setcounter{tocdepth}{2}


\begin{document}

\noindent
Teacher: Samuel Drapeau \hfill Shanghai Jiao Tong University \newline
Teaching Assistant: Zhang Yaoyuan \hfill WS 2015/2016

\smallskip
\noindent
\hrulefill

\smallskip
%-----------------------   mainmatter   ------------------------------------------

\setcounter{section}{5}

\pagestyle{empty}


%-----------------------------------------------------------------------
\section*{``Stochastic Processes'' -- Homework Sheet 6}
\thispagestyle{empty}


%-----------------------------------------------------------------------


Let $(\Omega,\mathcal{F},\mathbb{F},P)$ a filtrated probability space with $\mathbb{F}=(\mathcal{F}_t)_{t=0,1,\ldots}$.

\begin{exercise}(10 points)
    Let $X$ be a sub-martingale.
    Let $\sigma$ and $\tau$ be two stopping times, such that $\sigma\leq \tau \leq T$ for an integer $T$.
    We set
    \begin{equation*}
        \tau_0=0
    \end{equation*}
    and recursively
    \begin{align*}
        \tau_1(\omega) & = \inf\{t\colon \sigma(\omega)\leq t\leq \tau(\omega), t\geq \tau_0(\omega), X_t(\omega) < x\}\\
        \tau_2(\omega) & = \inf\{t\colon \sigma(\omega)\leq t\leq \tau(\omega), t\geq \tau_1(\omega), X_t(\omega)>y\}\\
        & \vdots\\
        \tau_{2k-1}(\omega) & = \inf\{t \colon \sigma(\omega)\leq t\leq \tau(\omega), t\geq \tau_{2k-2}(\omega), X_t(\omega) < x\}\\
        \tau_{2k}(\omega) & = \inf\{t \colon \sigma(\omega)\leq t\leq \tau(\omega), t\geq \tau_{2k-1}(\omega), X_t(\omega)>y\}
    \end{align*}
    with the convention that the infimum over the empty set is infinite.
    Define
    \begin{equation*}
        U_{\llbracket \sigma(\omega),\tau(\omega)\rrbracket}(x,y,X(\omega))=\sup\{k\colon \tau_{2k}(\omega)<\infty\}
    \end{equation*}
    where $\llbracket \sigma(\omega),\tau(\omega)\rrbracket=\{t \in \mathbb{N}\colon \sigma(\omega)\leq t\leq \tau(\omega)\}$.
    \begin{enumerate}[label=\textit{(\roman*)}]
        \item Show that
            \begin{equation*}
                \left( y-x \right)E\left[  U_{\llbracket \sigma,\tau\rrbracket}(x,y,X) |\mathcal{F}_{\sigma}\right]\leq E\left[ \left( X_{\tau}-x \right)^+ |\mathcal{F}_{\sigma}\right]-E\left[ \left( X_{\sigma}-x \right)^+ |\mathcal{F}_{\sigma}\right]
            \end{equation*}
        \item Show that
            \begin{equation*}
                P\left[ \sup_{t} X_t \geq \lambda \right]\leq \frac{1}{\lambda}\sup_t E\left[ X_t^+ \right]
            \end{equation*}
    \end{enumerate}
\end{exercise}
%\begin{proof}
    
%\end{proof}

%\begin{proof}
    
%\end{proof}

\begin{exercise}(20 points)

We consider the following random walk starting at $0$
\begin{equation*}
    S_0=0\quad \text{and}\quad S_t=\sum_{s=1}^t X_s,\quad t\geq 1
\end{equation*}
where 
\begin{equation*}
    X_t(\omega)=
    \begin{cases}
        1&\text{if } \omega_t=1,\\
        -1&\text{if }\omega_t=-1.
    \end{cases}
    \quad t\geq 1\text{ and }\omega=(\omega_t)_{t\in \mathbb{N}} \in \Omega =\{-1,1\}^{\mathbb{N}}.
\end{equation*}
As for the filtration we take
\begin{equation*}
    \mathcal{F}_0=\{\emptyset,\Omega\}\quad \text{and}\quad \mathcal{F}_t=\sigma(X_s\colon 1\leq s\leq t),\quad t\geq 1
\end{equation*}
On $\mathcal{F}=\otimes_{t \in \mathbb{N}} \{\emptyset, \{-1\},\{1\},\{-1,1\}\}$ and the unique probability on $\mathcal{F}$ which on the finite cylinder is given by
\begin{equation*}
    P[A]=p^lq^{t-l}
\end{equation*}
where $p\in [0,1]$, $q=1-p$ and
\begin{equation*}
    A=\{\omega \in \Omega\colon \omega_{t_k}=e_k, k=1,\ldots n\}
\end{equation*}
for the finite number of times $t_1,\ldots, t_n$, numbers $e_k\in \{-1,1\}$ and $l$ is equal to the numbers of those $e_k=1$.
In other words, this is the probability that we have head at time $t_k$ when $e_k=1$ and tail when $e_k=-1$ for the finite numbers of time $t_1,\ldots, t_n$ but with a biased coin which provides a probability $p$ of having head and $q=1-p$ of having tail.

\begin{enumerate}[label=\textit{(\roman*)}]
    \item Show that $S$ is a martingale if and only if $p=1/2$. 
        Show in that case that
        \begin{equation*}
            P\left[ \liminf X_t=-\infty\text{ and } \limsup X_t =\infty\right]=1
        \end{equation*}
    \item Show in the general case
        \begin{equation*}
            P\left[ \lim X_t=\infty \right]=1,\quad \text{if } p>1/2;
        \end{equation*}
        and
        \begin{equation*}
            P\left[ \lim X_t =-\infty \right]=1,\quad \text{if }p<1/2.
        \end{equation*}
        To do so, show first that $E[X_t|\mathcal{F}_{t-1}]=E[X_t]$ for every $t\geq 1$.
        Then, that the Doob-Meyer decomposition
        \begin{equation*}
            S=M-A,
        \end{equation*}
        is such that $A_t=-t(2p-1)$ and $M$ satisfies
        \begin{equation*}
            P\left[ \liminf M_t=-\infty\text{ and }\limsup M_t=\infty \right]=1
        \end{equation*}
        and use exercise 3.1.
    \item Suppose that $p=1/2$.
        Define $\tau =\inf\{t\colon S_t=a\text{ or }S_t=-b\}$ for two integers $a,b$.
        \begin{itemize}
            \item Show that $P[\tau<\infty]=1$.
            \item Show that
                \begin{equation*}
                    P\left[ S_{\tau}=a \right]=\frac{b}{a+b}
                \end{equation*}
                This is the probability that $S$ reach the value $a$ before hitting $-b$.
            \item Show that $M=S^2_t-t$ is a martingale.
                Deduce that
                \begin{equation*}
                    E\left[ \tau \right]=ab
                \end{equation*}
        \end{itemize}
    \item Suppose that $p\neq 1/2$, in the general case, show that $M_t:=(q/p)^{S_t}$ is a martingale.
        Define $\tau =\inf\{t\colon S_t=a\text{ or }S_t=-b\}$ for two integers $a,b$.
        Show that
        \begin{equation*}
            P[S_{\tau}=a]=\frac{(q/p)^b-1}{(q/p)^{a+b}-1}
        \end{equation*}
\end{enumerate}
\end{exercise}
%\begin{proof}
    
%\end{proof}


\smallskip
\noindent
\textbf{Due date:} Upload before Monday 2015.11.02 14:00.

\end{document}
