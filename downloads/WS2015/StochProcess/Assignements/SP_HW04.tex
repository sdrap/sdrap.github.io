
%-----------------------   general info   ----------------------------------------	

%	filename  	= 	homework_template
%   date      	= 	
%   time      	= 	
%   author    	= 	Samuel Drapeau (Based on a template provided by Daniel, University of Constance)
% 	adress		= 	Shanghai Jiao Tong University

%-----------------------   documentclass, packages   -----------------------------
\documentclass[DIV=classic,a4paper,10pt]{scrartcl}

\KOMAoptions{DIV=last}

\setkomafont{title}{\scshape}
\setkomafont{disposition}{\normalcolor\scshape}

%\setkomafont{title}{\bfseries}
%\setkomafont{disposition}{\normalfont\normalcolor\bfseries}


%----------------------- general packages (fonts, language, enumitem) ------------
\usepackage{times}
\usepackage[english]{babel}
\usepackage[T1]{fontenc}
\usepackage[latin1]{inputenc}
\usepackage{enumitem}
%\usepackage{fullpage}



%----------------------- Math Packages -------------------------------------------

\usepackage{amsfonts}
\usepackage{amssymb}
\usepackage[intlimits]{amsmath}
\usepackage[hyperref,amsmath,thmmarks]{ntheorem}

%\parindent0cm

%-----------------------   math operators   --------------------------------------	
%-----------------------   general short cuts   ----------------------------------	

%-----------------------   theorem environments   --------------------------------


\theoremseparator{.}
\newtheorem{theorem}{Theorem}[section]
\newtheorem{corollary}[theorem]{Corollary}
\newtheorem{conjecture}[theorem]{Conjecture}
\newtheorem{assumption}[theorem]{Assumption}
\newtheorem{lemma}[theorem]{Lemma}
\newtheorem{proposition}[theorem]{Proposition}
\newtheorem{exercise}[theorem]{Exercise}

\theorembodyfont{\upshape}
\newtheorem{definition}[theorem]{Definition}

\theoremsymbol{\ensuremath{\lozenge}}
\newtheorem{example}[theorem]{Example}

\theoremsymbol{\ensuremath{\blacklozenge}}
\theoremheaderfont{\itshape}
\newtheorem{remark}[theorem]{Remark}

\theoremsymbol{\ensuremath{\blacklozenge}}
\theoremheaderfont{\itshape}
\newtheorem{remarks}[theorem]{Remarks}

\theoremsymbol{\ensuremath{\square}}
\theoremheaderfont{\itshape}
\theoremstyle{nonumberplain}
\newtheorem{proof}{Proof}



%---------------------------------------------------------------------------------
% \numberwithin{section}{chapter}\numberwithin{equation}{chapter}
\numberwithin{equation}{section}
\setcounter{tocdepth}{2}


\begin{document}

\noindent
Teacher: Samuel Drapeau \hfill Shanghai Jiao Tong University \newline
Teaching Assistant: Zhang Youyuan \hfill WS 2015/2016

\smallskip
\noindent
\hrulefill

\smallskip
%-----------------------   mainmatter   ------------------------------------------

\setcounter{section}{4}

\pagestyle{empty}


%-----------------------------------------------------------------------
\section*{``Stochastic Processes'' -- Homework Sheet 4}
\thispagestyle{empty}


%-----------------------------------------------------------------------
Throughout, $(\Omega, \mathcal{F}, \mathbb{F},P)$ is a filtrated probability space where $\mathbf{T}=\mathbb{N}_0$.




\begin{exercise}(10 points)
    \begin{enumerate}[label=\textit{(\alph*)}]
        \item Let $X$ and $Y$ be two super-martingales.
            Show that $X\wedge Y$ is a super-martingale;
        \item Let $X$ be a predictable martingale, show that $X_t=X_0$ for every $t$.
        \item Let $\xi \in L^1$, show that $X_t:=E[\xi |\mathcal{F}_t]$ defines a martingale $X$;
        \item Show, or find a counter example for the following assertion: Let $X$ be an adapted process such that $X_t$ is integrable and $E[X_0]=E[X_t]$ for every $t$. Then $X$ is a martingale.
    \end{enumerate}
\end{exercise}


\begin{exercise}(5 points)
    Let $I\subseteq \mathbf{T}$ be such that $\sup I=\infty$.
    Show that every super-martingale $X$ such that $E[X_0]\leq E[X_t]$ for every $t \in I$ is a martingale.
\end{exercise}




\begin{exercise}(10 points)
    Let $X$ be an adapted process such that $E[\sup_t |X_t|]<\infty$.
    Denote by $\mathcal{T}$ the set of all stopping times and let $T \in \mathbb{N}_0$ be an arbitrary time horizon.
    We define recursively
    \begin{equation*}
        S_T=X_T\quad \text{and}\quad S_t=\max \left\{ E\left[ S_{t+1}|\mathcal{F}_t \right];X_t \right\}, \quad t\leq T-1.
    \end{equation*}
    Define further $\tau^t=\inf\{s\colon s\geq t \text{ and }S_s=X_s\}$ for every $t=0,\ldots, T$.
    \begin{enumerate}[label=(\alph*)]
        \item Show that $S=(S_t)_{t=0,\ldots, T}$ is a super-martingale such that $S_t\geq X_t$ for every $t=0,\ldots, T$;
        \item Let $U=(U_t)_{t=0,\ldots,T}$ be a super-martingale such that $U_t\geq X_t$ for every $t=0,\ldots,T$.
            Show that $U_t\geq X_t$ for every $t=0,\ldots, T$;
        \item Show that $E[X_{\tau^t}|\mathcal{F}_s]=E[S_t|\mathcal{F}_s]$ for every $s\leq t\leq T$ and conclude that
            \begin{equation*}
                E\left[ X_{\tau^t} \right]=E\left[ S_t \right]=\max_{\{\tau \in \mathcal{T}\colon t\leq \tau\leq T\}} E\left[ X_\tau \right]
            \end{equation*}
        \item We denote by $S^T=(S_t^T)_{t=0,\ldots, T}$ the process defined in (a) whereby, we stress the dependence on the time horizon due to its recursive definition.
            Clearly, $S_t=\lim_{T\to \infty} S^T_t$ defines a process $S$.
            Show that $S$ is a super-martingale for which holds $S\geq X$.
            Show further that for every other super-martingale $U$ such that $U\geq X$ it follows $U\geq S$.
    \end{enumerate}
\end{exercise}
\begin{exercise}(Bonus 10 points)
    Consider now our example of coin tossing but infinitely many times.
    As seen, the state  space is defined as follows
    \begin{equation*}
        \Omega=\prod_{t \in \mathbb{N}} \{-1,1\}=\{-1,1\}^{\mathbb{N}}=\{\omega=(\omega_t)\colon \omega_t =\pm 1\text{ for every }t\}
    \end{equation*}
    On each $\Omega_t=\{-1,1\}$ we consider the $\sigma$-algebra $\mathcal{F}_t=\{\emptyset,\{-1\},\{1\}, \{-1,1\}\}$ and on $\Omega$ the product $\sigma$-algebra $\mathcal{F}=\otimes\mathcal{F}_t$.
    \begin{enumerate}[label=(\alph*)]
        \item Show that the collection $\mathcal{R}$ of finite product cylinders
            \begin{equation}\label{eq:finite_cylinder_cointoss}
                A=\{\omega=(\omega_t) \in \Omega \colon \omega_{t_k}=e_k, k=1,\ldots n\}
            \end{equation}
            for a given set of values $e_k\in \{-1,1\}$, and times $t_k \in \mathbb{N}$, $k=1,\ldots,n$, together with the empty-set is a semi-ring.
        \item For $p\in [0,1]$, we define $P:\mathcal{R}\to [0,1]$ as follows
            \begin{equation*}
                P[\emptyset]=0\quad \text{ and }\quad P[A]=p^{l}(1-p)^{n-l}
            \end{equation*}
            for every $A \in \mathcal{R}$ of the form \eqref{eq:finite_cylinder_cointoss} where $l$ is equal to the number of those $k=1,\ldots,n$ where $e_k=1$.
            Show that $P$ defines a content.
        \item Show that for every finite family $(A_k)_{k\leq n}$ of elements in $\mathcal{R}$ and $A\in \mathcal{R}$ such that $A\subseteq \cup_{k\leq n}A_k$, it holds\footnote{Consider first the case where $n=2$ and $A_1,A_2$ are one dimensional product cylinder.}
            \begin{equation*}
                P[A]\leq \sum_{k\leq n}P[A_k].
            \end{equation*}
        \item Show that for every countable family $(A_n)$ of elements in $\mathcal{R}$ and $A\in \mathcal{R}$ such that $A\subseteq \cup A_n$, it holds\footnote{Show that there exists a finite $n_0$ such that $A\subseteq \cup_{k\leq n_0}A_k$.}
            \begin{equation*}
                P[A]\leq \sum P[A_n],
            \end{equation*}
            and deduce using Caratheodory's theorem that $P$ extends uniquely to a probability measure $P$ on $\mathcal{F}$.
        \item Defining the process $X$ by $X_t(\omega)=\omega_t$, describe the filtration generated by $X$.
    \end{enumerate}
\end{exercise}


\smallskip
\noindent
\textbf{Due date:} Upload before Monday 2015.10.26 14:00.

\end{document}
