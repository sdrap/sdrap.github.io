%-----------------------   general info   ----------------------------------------	

%	filename  	= 	homework_template
%   date      	= 	
%   time      	= 	
%   author    	= 	Samuel Drapeau (Based on a template provided by Daniel, University of Constance)
% 	adress		= 	Shanghai Jiao Tong University

%-----------------------   documentclass, packages   -----------------------------
\documentclass[DIV=classic,a4paper,10pt]{scrartcl}

\KOMAoptions{DIV=last}

\setkomafont{title}{\scshape}
\setkomafont{disposition}{\normalcolor\scshape}

%\setkomafont{title}{\bfseries}
%\setkomafont{disposition}{\normalfont\normalcolor\bfseries}


%----------------------- general packages (fonts, language, enumitem) ------------
\usepackage{times}
\usepackage[english]{babel}
\usepackage[T1]{fontenc}
\usepackage[latin1]{inputenc}
\usepackage{enumitem}
%\usepackage{fullpage}



%----------------------- Math Packages -------------------------------------------

\usepackage{amsfonts}
\usepackage{amssymb}
\usepackage[intlimits]{amsmath}
\usepackage[hyperref,amsmath,thmmarks]{ntheorem}
\usepackage{stmaryrd}
%\parindent0cm

%-----------------------   math operators   --------------------------------------	
%-----------------------   general short cuts   ----------------------------------	

%-----------------------   theorem environments   --------------------------------


\theoremseparator{.}
\newtheorem{theorem}{Theorem}[section]
\newtheorem{corollary}[theorem]{Corollary}
\newtheorem{conjecture}[theorem]{Conjecture}
\newtheorem{assumption}[theorem]{Assumption}
\newtheorem{lemma}[theorem]{Lemma}
\newtheorem{proposition}[theorem]{Proposition}
\newtheorem{exercise}[theorem]{Exercise}

\theorembodyfont{\upshape}
\newtheorem{definition}[theorem]{Definition}

\theoremsymbol{\ensuremath{\lozenge}}
\newtheorem{example}[theorem]{Example}

\theoremsymbol{\ensuremath{\blacklozenge}}
\theoremheaderfont{\itshape}
\newtheorem{remark}[theorem]{Remark}

\theoremsymbol{\ensuremath{\blacklozenge}}
\theoremheaderfont{\itshape}
\newtheorem{remarks}[theorem]{Remarks}

\theoremsymbol{\ensuremath{\square}}
\theoremheaderfont{\itshape}
\theoremstyle{nonumberplain}
\newtheorem{proof}{Proof}



%---------------------------------------------------------------------------------
% \numberwithin{section}{chapter}\numberwithin{equation}{chapter}
\numberwithin{equation}{section}
\setcounter{tocdepth}{2}


\begin{document}

\noindent
Teacher: Samuel Drapeau \hfill Shanghai Jiao Tong University \newline
Teaching Assistant: Zhang Yaoyuan \hfill WS 2015/2016

\smallskip
\noindent
\hrulefill

\smallskip
%-----------------------   mainmatter   ------------------------------------------

\setcounter{section}{7}

\pagestyle{empty}


%-----------------------------------------------------------------------
\section*{``Stochastic Processes'' -- Homework Sheet 7}
\thispagestyle{empty}


%-----------------------------------------------------------------------


\begin{exercise}(5 points)
    
    Let $(\Omega,\mathcal{F},P)$ be a probability space.
    Let $X=(X_t)_{t\in \mathbb{N}}$ be a process of integrable independent random variables, that is
    \begin{equation*}
        P\left[ X_{t_1}\in A_1,\ldots,X_{t_n}\in A_n \right]=\prod_{k=1}^nP[X_{t_k}\in A_k]
    \end{equation*}
    for every finite subset of times $t_1,\ldots t_n$ and Borel sets $A_1,\ldots A_n$.
    Suppose that $E[X_t]=1$ for every $t$.

    For the filtration
    \begin{equation*}
        \mathcal{F}_t=\sigma(X_s\colon s\leq t), 
    \end{equation*}
    Show that the process $Y$ given by
    \begin{equation*}
        Y_t=\prod_{s=0}^t X_s, \quad s\geq 0
    \end{equation*}
    is a martingale.
\end{exercise}


\begin{exercise}(10 points)

    In this exercise, we will consider a simple discrete price process model with finite horizon $T \in \mathbb{N}$.
    
    We denote by $S=(S_{t})_{0\leq t\leq T}$ the evolution of a stock price for the times $t=0,\ldots ,T$.
    Usually, stock prices are strictly positives and characterized by their returns
    \begin{equation*}
        R_t = \frac{S_{t}-S_{t-1}}{S_t}, \quad 1\leq t\leq T
    \end{equation*}
    which is the proportional gain/loss of the price evolution between $t-1$ and $t$.
    Or in other terms, if the returns $R=(R_t)_{1\leq t\leq T}$ are given, the stock price is then
    \begin{equation*}
        S_t=S_0\prod_{s=1}^t\left( 1+R_s \right), \quad 0\leq t\leq T
    \end{equation*}
    for a given start price $S_0>0$.
    To guarantee that the stock price remains strictly positive, we assume that $R_t>-1$ for every $t=1,\ldots T$.

    Our simple model is as follows.
    Let 
    \begin{itemize}
        \item $\Omega =\{-1,1\}^T=\{\omega=(\omega_t)_{1\leq t\leq T}\colon \omega_t \in \{-1,1\}\}$;
        \item For $-1<d<u$, (where $d$ stands for down and $u$ for up) we define
            \begin{equation*}
                R_t(\omega):= \
                \begin{cases}
                    u&\text{if }\omega_t=1\\
                    d&\text{if }\omega_t=-1
                \end{cases},\quad 1\leq t\leq T
            \end{equation*}
        \item As for the filtration we take
            \begin{equation*}
                \mathcal{F}_0=\{\emptyset,\Omega\}\quad \text{and}\quad \mathcal{F}_t=\sigma(R_s\colon 1\leq s\leq t),\quad 1\leq t\leq T
            \end{equation*}
    \end{itemize}
    Show that if $-1<d<0<u$, then there exists a unique probability measure $P$ on $2^{\Omega}$ such that $S$ is a martingale.
    Hereby, show that this probability measure is such that $P[\omega_t=1]=-d/(u-d)$.
    Show that under this measure, the random variables $R_1,\ldots, R_T$ are independent.
\end{exercise}

\begin{exercise}(Bonus 10 points)
    Find a probability space, $(\Omega, \mathcal{F},P)$, a filtration $\mathbb{F}=(\mathcal{F}_t)_{t\in \mathbb{N}}$, a process $X$ and a random variable $X_T$ such that
    \begin{itemize}
        \item $X$ is a martingale;
        \item $\sup_t E[|X_t|]<\infty$;
        \item $X_t\to X_T$ $P$-almost surely;
        \item but $E[|X_t-X|]\not \to 0$.
    \end{itemize}
\end{exercise}



%\begin{proof}
    
%\end{proof}


\smallskip
\noindent
\textbf{Due date:} Upload before Monday 2015.11.09 14:00.

\end{document}
