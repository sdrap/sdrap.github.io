
%-----------------------   general info   ----------------------------------------	

%	filename  	= 	homework_template
%   date      	= 	
%   time      	= 	
%   author    	= 	Samuel Drapeau (Based on a template provided by Daniel, University of Constance)
% 	adress		= 	Shanghai Jiao Tong University

%-----------------------   documentclass, packages   -----------------------------
\documentclass[DIV=classic,a4paper,10pt]{scrartcl}

\KOMAoptions{DIV=last}

\setkomafont{title}{\scshape}
\setkomafont{disposition}{\normalcolor\scshape}

%\setkomafont{title}{\bfseries}
%\setkomafont{disposition}{\normalfont\normalcolor\bfseries}


%----------------------- general packages (fonts, language, enumitem) ------------
\usepackage{times}
\usepackage[english]{babel}
\usepackage[T1]{fontenc}
\usepackage[latin1]{inputenc}
\usepackage{enumitem}
%\usepackage{fullpage}



%----------------------- Math Packages -------------------------------------------

\usepackage{amsfonts}
\usepackage{amssymb}
\usepackage[intlimits]{amsmath}
\usepackage[hyperref,amsmath,thmmarks]{ntheorem}

%\parindent0cm

%-----------------------   math operators   --------------------------------------	
%-----------------------   general short cuts   ----------------------------------	

%-----------------------   theorem environments   --------------------------------


\theoremseparator{.}
\newtheorem{theorem}{Theorem}[section]
\newtheorem{corollary}[theorem]{Corollary}
\newtheorem{conjecture}[theorem]{Conjecture}
\newtheorem{assumption}[theorem]{Assumption}
\newtheorem{lemma}[theorem]{Lemma}
\newtheorem{proposition}[theorem]{Proposition}
\newtheorem{exercise}[theorem]{Exercise}

\theorembodyfont{\upshape}
\newtheorem{definition}[theorem]{Definition}

\theoremsymbol{\ensuremath{\lozenge}}
\newtheorem{example}[theorem]{Example}

\theoremsymbol{\ensuremath{\blacklozenge}}
\theoremheaderfont{\itshape}
\newtheorem{remark}[theorem]{Remark}

\theoremsymbol{\ensuremath{\blacklozenge}}
\theoremheaderfont{\itshape}
\newtheorem{remarks}[theorem]{Remarks}

\theoremsymbol{\ensuremath{\square}}
\theoremheaderfont{\itshape}
\theoremstyle{nonumberplain}
\newtheorem{proof}{Proof}



%---------------------------------------------------------------------------------
% \numberwithin{section}{chapter}\numberwithin{equation}{chapter}
\numberwithin{equation}{section}
\setcounter{tocdepth}{2}


\begin{document}

\noindent
Teacher: Samuel Drapeau \hfill Shanghai Jiao Tong University \newline
Teaching Assistant: Zhang Youyuan \hfill WS 2015/2016

\smallskip
\noindent
\hrulefill

\smallskip
%-----------------------   mainmatter   ------------------------------------------

\setcounter{section}{1}

\pagestyle{empty}


%-----------------------------------------------------------------------
\section*{``Stochastic Processes'' -- Homework Sheet 1}
\thispagestyle{empty}


%-----------------------------------------------------------------------

\begin{exercise} (12 Points) 
    \begin{enumerate}[label=(\alph*), fullwidth]
        \item Let $\Omega$ be a state space, and let $(\mathcal{F}_i)$ be an arbitrary family of $\sigma$-algebra on $\Omega$.
            Show that 
            \begin{equation*}
              \mathcal{F}=\bigcap \mathcal{F}_i=\left\{A \subseteq \Omega\colon A \in \mathcal{F}_i \text{ for all }i\right\},
            \end{equation*}
            is a $\sigma$-algebra.
            Conclude that for a collection $\mathcal{C}$ of subsets of $\Omega$,
            \begin{equation*}
                \sigma\left(\mathcal{C}\right):=\bigcap\left\{\mathcal{F}\colon \mathcal{F}\text{ is a }\sigma\text{-algebra with } \mathcal{C}\subseteq \mathcal{F}\right\},
            \end{equation*}
            is the unique smallest $\sigma$-algebra containing $\mathcal{C}$.
        \item Give an example where the union of two $\sigma$-algebras is not a $\sigma$-algebra.
        \item 
            Let $\Omega=\mathbb{R}$, and $\mathcal{F}=\mathcal{B}(\mathbb{R})$ the Borel $\sigma$-algebra of $\mathbb{R}$, that is, the $\sigma$-algebra generated by the collection $\{O\colon O \text{ open set in }\mathbb{R}\}$.
            It holds that $\mathcal{B}(\mathbb{R})=\sigma(\mathcal{C}_i)$ for each $i=0,\ldots,17$ where
            \begin{align*}
                \mathcal{C}_0&=\left\{O\colon O\text{ open subset of }\mathbb{R}\right\} &  \mathcal{C}_1&=\left\{F\colon F\text{ closed subset of }\mathbb{R}\right\} \\
                \mathcal{C}_2&=\left\{]a,b[\colon a\leq b \text{ with } a,b\in \mathbb{R}\right\} & \mathcal{C}_3&=\left\{[a,b]\colon a\leq b \text{ with } a,b\in \mathbb{R}\right\}\\
                \mathcal{C}_4&=\left\{]a,b]\colon a\leq b \text{ with } a,b\in \mathbb{R}\right\} & \mathcal{C}_5&=\left\{[a,b[\colon a\leq b \text{ with } a,b\in \mathbb{R}\right\}\\
                \mathcal{C}_6&=\left\{]-\infty,b]\colon b\in \mathbb{R}\right\} & \mathcal{C}_7&=\left\{]-\infty,b[\colon b\in \mathbb{R}\right\}\\
                \mathcal{C}_8&=\left\{[a,\infty[\colon a\in \mathbb{R}\right\} & \mathcal{C}_9&=\left\{]a,\infty[\colon a\in \mathbb{R}\right\}\\
                \mathcal{C}_{10}&=\left\{]a,b[\colon a\leq b \text{ with } a,b\in \mathbb{Q}\right\} & \mathcal{C}_{11}&=\left\{[a,b]\colon a\leq b \text{ with } a,b\in \mathbb{Q}\right\}\\
                \mathcal{C}_{12}&=\left\{]a,b]\colon a\leq b \text{ with } a,b\in \mathbb{Q}\right\} & \mathcal{C}_{13}&=\left\{[a,b[\colon a\leq b \text{ with } a,b\in \mathbb{Q}\right\}\\
                \mathcal{C}_{14}&=\left\{]-\infty,b]\colon b\in \mathbb{Q}\right\} & \mathcal{C}_{15}&=\left\{]-\infty,b[\colon b\in \mathbb{Q}\right\}\\
                \mathcal{C}_{16}&=\left\{[a,\infty[\colon a\in \mathbb{Q}\right\} & \mathcal{C}_{17}&=\left\{]a,\infty[\colon a\in \mathbb{Q}\right\}\\
            \end{align*}

            By definition $\mathcal{B}(\mathbb{R})=\sigma(\mathcal{C}_0)$.
            Show the assertion for the cases $i=1,9$ and $12$.
    \end{enumerate}
\end{exercise}
\begin{proof}
    
\end{proof}
\begin{exercise} (12 Points)
    \begin{enumerate}[label=(\alph*), fullwidth]

        \item Let $(\Omega,\mathcal{F})$ and $(S,\mathcal{S})$ be two measurable spaces.
            Given a function $X:\Omega \to S$, show that the collection of sets
            \begin{equation*}
                \sigma(X):=\left\{ X^{-1}(B)=\{\omega \in \Omega\colon X(\omega)\in B\}\colon B\in \mathcal{S} \right\},
            \end{equation*}
            is a $\sigma$-algebra on $\Omega$.

            Give a simple example where 
            \begin{equation*}
                \left\{  X(A)=\{X(\omega)\colon \omega \in A\}\colon A\in \mathcal{F}\right\}
            \end{equation*}
            is not a $\sigma$-algebra on $S$.
        \item Let $(\Omega,\mathcal{F})$, $(S,\mathcal{S})$ and $(T,\mathcal{T})$ be three measurable spaces.
            Given a $\mathcal{F}$-$\mathcal{S}$-measurable function $X:\Omega \to S$ and a $\mathcal{S}$-$\mathcal{T}$-measurable function $Y:S\to T$, show that $Z=Y\circ X:\Omega \to T$ is a $\mathcal{S}$-$\mathcal{T}$- measurable function.

        \item Let $(\Omega, \mathcal{F})$ be a measurable space, and $X,Y$ be random variables as well as $(X_n)$ be a sequence of random variables.
            Show that
            \begin{itemize}
                \item $aX+bY$ is a random variable for every $a,b \in \mathbb{R}$;
                \item $XY$ is a random variable;
                \item $\max(X,Y)$ and $\min(X,Y)$ are random variables;
                \item $\sup X_n$ and $\inf X_n$ are extended real valued random variables;\footnote{With respect to the Borel $\sigma$-algebra on $[-\infty.\infty]$ generated by the metric $d(x,y)=|\arctan(x)-\arctan(y)|$ that coincide with the euclidean topology on $\mathbb{R}$. An extended random variable $X:\Omega \to [-\infty,\infty]$ is measurable if and only if $\{X\leq t\} \in \mathcal{F}$ for every $t \in \mathbb{R}$.}
                \item $\liminf X_n:=\inf_n\sup_{k\geq n}X_k$ and $\limsup X_n:=\inf_n\sup_{k\geq n}X_k$ are extended real valued random variables;
                \item $A:=\{\lim X_n \text{ exists}\}:=\{\omega \colon \lim X_n(\omega)\text{ exists}\}=\{\liminf X_n=\limsup X_n\}$ is measurable.
            \end{itemize}
            %\textit{For the first three points, use the fact that continuous functions from $\mathbb{R}^2$ to $\mathbb{R}$ are measurable, and that $\mathcal{B}(\mathbb{R}\times \mathbb{R})$ is generated by sets of the form $A\times B$ for $A,B \in \mathcal{B}(\mathbb{R})$.}
    \end{enumerate}
\end{exercise}
\begin{proof}
    
\end{proof}

\begin{exercise} (12 Points) \newline
    Let $(\Omega,\mathcal{F},P)$ be a probability space.
    Let $X:\Omega \to \mathbb{R}$ be a random variable.
    Define
    \begin{equation*}
        F(t):=P[X\leq t], \quad t \in \mathbb{R}.
    \end{equation*}
    which is called the \emph{cumulative distribution function of }$X$.
    Show that
    \begin{enumerate}[label=\arabic*)]
        \item $F:\mathbb{R}\to \mathbb{R}$ is increasing, $\lim_{t \to -\infty}F(t)=0$, $\lim_{t \to \infty}F(t)=1$ and $F$ is right-continuous.\footnote{That is $\lim_{s\nearrow t}F(s)=F(t)$.}
        \item $F$ is measurable for the Borel $\sigma$-algebra;
        \item $F$ has at most countably many discontinuous points.
    \end{enumerate}
\end{exercise}
\begin{proof}
    
\end{proof}

\begin{exercise} (Bonus 12 Points) \newline
    Let $(\Omega,\mathcal{F},P)$ be a probability space and let $(A_i)_{i\in I}$ be an arbitrary family -- not necessarily countable -- of measurable sets.
    Suppose that $P[A_i\cap A_j]=0$ for every $i,j\in I$ with $i\neq j$ and $P[A_i]>0$ for every $i\in I$.
    Show that the family $(A_i)_{i\in I}$ is at most countable.
\end{exercise}
\begin{proof}
    
\end{proof}
\smallskip
\noindent
\textbf{Due date:} Upload before Monday 2015.09.28 14:00.

\end{document}
