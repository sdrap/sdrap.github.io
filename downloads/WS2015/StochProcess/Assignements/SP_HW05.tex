
%-----------------------   general info   ----------------------------------------	

%	filename  	= 	homework_template
%   date      	= 	
%   time      	= 	
%   author    	= 	Samuel Drapeau (Based on a template provided by Daniel, University of Constance)
% 	adress		= 	Shanghai Jiao Tong University

%-----------------------   documentclass, packages   -----------------------------
\documentclass[DIV=classic,a4paper,10pt]{scrartcl}

\KOMAoptions{DIV=last}

\setkomafont{title}{\scshape}
\setkomafont{disposition}{\normalcolor\scshape}

%\setkomafont{title}{\bfseries}
%\setkomafont{disposition}{\normalfont\normalcolor\bfseries}


%----------------------- general packages (fonts, language, enumitem) ------------
\usepackage{times}
\usepackage[english]{babel}
\usepackage[T1]{fontenc}
\usepackage[latin1]{inputenc}
\usepackage{enumitem}
%\usepackage{fullpage}



%----------------------- Math Packages -------------------------------------------

\usepackage{amsfonts}
\usepackage{amssymb}
\usepackage[intlimits]{amsmath}
\usepackage[hyperref,amsmath,thmmarks]{ntheorem}

%\parindent0cm

%-----------------------   math operators   --------------------------------------	
%-----------------------   general short cuts   ----------------------------------	

%-----------------------   theorem environments   --------------------------------


\theoremseparator{.}
\newtheorem{theorem}{Theorem}[section]
\newtheorem{corollary}[theorem]{Corollary}
\newtheorem{conjecture}[theorem]{Conjecture}
\newtheorem{assumption}[theorem]{Assumption}
\newtheorem{lemma}[theorem]{Lemma}
\newtheorem{proposition}[theorem]{Proposition}
\newtheorem{exercise}[theorem]{Exercise}

\theorembodyfont{\upshape}
\newtheorem{definition}[theorem]{Definition}

\theoremsymbol{\ensuremath{\lozenge}}
\newtheorem{example}[theorem]{Example}

\theoremsymbol{\ensuremath{\blacklozenge}}
\theoremheaderfont{\itshape}
\newtheorem{remark}[theorem]{Remark}

\theoremsymbol{\ensuremath{\blacklozenge}}
\theoremheaderfont{\itshape}
\newtheorem{remarks}[theorem]{Remarks}

\theoremsymbol{\ensuremath{\square}}
\theoremheaderfont{\itshape}
\theoremstyle{nonumberplain}
\newtheorem{proof}{Proof}



%---------------------------------------------------------------------------------
% \numberwithin{section}{chapter}\numberwithin{equation}{chapter}
\numberwithin{equation}{section}
\setcounter{tocdepth}{2}


\begin{document}

\noindent
Teacher: Samuel Drapeau \hfill Shanghai Jiao Tong University \newline
Teaching Assistant: Zhang Yaoyuan \hfill WS 2015/2016

\smallskip
\noindent
\hrulefill

\smallskip
%-----------------------   mainmatter   ------------------------------------------

\setcounter{section}{5}

\pagestyle{empty}


%-----------------------------------------------------------------------
\section*{``Stochastic Processes'' -- Homework Sheet 5}
\thispagestyle{empty}


%-----------------------------------------------------------------------


Let $(\Omega,\mathcal{F},\mathbb{F},P)$ a filtrated probability space with $\mathbb{F}=(\mathcal{F}_t)_{t=0,1,\ldots}$.

\begin{exercise}(10 points)

    Provide an example of a martingale $X=(X_t)$ such that $\sup_t E[|X_t|]<\infty$ and $X_t\to X_{\infty}$ $P$-almost surely for some $X_{\infty}$ but for which however it does not hold $E[|X_t-X_{\infty}|] \to 0$.
\end{exercise}
%\begin{proof}
    
%\end{proof}

\begin{exercise}(10 points)

    Let $X$ be an adapted and integrable stochastic process.
    Show that if $E[X_\tau]=E[X_0]$ for every finite stopping time\footnote{That is $P[\tau\leq T]=1$ for some $T \in \mathbb{N}$} $\tau$ then $X$ is a martingale.
\end{exercise}

%\begin{proof}
    
%\end{proof}

\begin{exercise}(20 points)

    Under the same assumptions as Exercise 4.3 of the Homework sheet 4, that is, let $X$ be an adapted process such that $E[\sup_t |X_t|]<\infty$.
    For the questions a) to d), we assume that $\mathbf{T}=\{0,\ldots,T\}$ for a given time horizon $T \in\mathbb{N}$.
    Denote by $\mathcal{T}$ the set of all stopping times with values in $\mathbf{T}$.
    We define recursively
    \begin{equation*}
        S_T=X_T\quad \text{and}\quad S_t=\max \left\{ E\left[ S_{t+1}|\mathcal{F}_t \right];X_t \right\}, \quad t\leq T-1.
    \end{equation*}
    and denote by $S=M-A$ the Doob decomposition of $S$ where $M$ is a martingale and $A$ is a predictable and integrable process with $A_0=0$.
    We define
    \begin{equation*}
        \tau_0=\inf\left\{ t\colon X_t=S_t \right\}\quad \text{and}\quad \tau_1=\inf\left\{ t\colon A_{t+1}>0 \right\}\wedge T
    \end{equation*}
    \begin{enumerate}[label=\text{\alph*)}, fullwidth]
        \item Show that $\tau_1$ is a stopping time such that $\tau_0\leq \tau_1$;
        \item Show that for every $\sigma \in \mathcal{T}$ the following assertions are equivalent:
            \begin{enumerate}[label=\textit{(\roman*)}]
                \item $E[X_\sigma]=\sup_{\tau \in \mathcal{T}}E[X_\tau]$;
                \item $X_\sigma=S_{\sigma}$ and $S^\sigma$ is a martingale;
                \item $\tau_0\leq \sigma\leq \tau_1$ and $E[X_{\sigma}]=E[S_{\sigma}]$.
            \end{enumerate}
        \item Let $\mathcal{M}_0$ be the set of all martingale $Y=(Y_t)$ such that $Y_0=0$.
            Show that
            \begin{equation*}
                \max_{\tau \in \mathcal{T}}E\left[ X_\tau \right]=\min_{Y \in \mathcal{M}_0}E\left[ \max_{0\leq t\leq T } \left( X_t-Y_t \right)\right]
            \end{equation*}
        \item Suppose that $T=2$, $\Omega=\{\omega_1,\omega_2,\omega_3,\omega_4\}$, $\mathcal{F}_0=\{\emptyset,\Omega\}$, $\mathcal{F}_1=\{\emptyset, \Omega, \{\omega_1,\omega_2\}, \{\omega_3,\omega_4\}\}$ and $\mathcal{F}_2=2^\Omega$.
            The process $X$ is given by
            \begin{equation*}
                X_0\equiv 10,\quad
                X_1(\omega)=
                \begin{cases}
                    5 &\text{if }\omega \in \{\omega_1,\omega_2\}\\
                    30 &\text{otherwise }
                \end{cases},\quad
                X_3(\omega)=
                \begin{cases}
                   5  & \text{if } \omega=\omega_1 \\
                   15 & \text{if } \omega=\omega_2 \\
                   10 & \text{if } \omega=\omega_3 \\
                   30 & \text{if } \omega=\omega_4 
                \end{cases}
            \end{equation*}
            compute $\tau_0$ and $\tau_1$.
        \item From now on, we consider that $\mathbf{T}=\mathbb{N}_0$, $T=\infty$ and consider the process $S$ defined in homework sheet 4.3 d).
            For $n \in \mathbb{N}$ we define $\mathcal{T}^n=\{\tau\in \mathcal{T}\colon n\leq \tau <\infty\}$ and $\tau_n=\inf\{t\colon n \leq t \text{ and }S_t=X_t\}$.
            Show that if $\tau_n <\infty$, then
            \begin{equation*}
                E\left[ X_{\tau_n} \right]=E\left[ S_n \right]=\max_{\tau \in \mathcal{T}^n}E\left[ X_{\tau} \right]
            \end{equation*}
    \end{enumerate}
\end{exercise}
%\begin{proof}
    
%\end{proof}


\smallskip
\noindent
\textbf{Due date:} Upload before Monday 2015.11.02 14:00.

\end{document}
